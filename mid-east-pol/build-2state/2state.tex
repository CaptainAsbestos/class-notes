\documentclass[letterpaper,12pt,twoside]{article} %define template defaults, paper size, main font, \&c.
  \usepackage[T1]{fontenc} %remove T2A to use other roman fonts, also have to remove all cyrillic text

%%Fonts \& Typesetting
  \usepackage{microtype} %pretty kerning
  \usepackage{dirtytalk} %easier to do quotation marks
  \usepackage{setspace} %setspace and setstrecth
  \usepackage{lipsum} %generates lorem ipsum
  %\usepackage{blindtext}
  \usepackage{mathptmx}
  \usepackage[super]{nth}

%%Page Geometry
\usepackage[top=1in, bottom=1in, left=0.75in, right=0.75in,headsep=0.2in]{geometry}
  \usepackage{multicol}
  \setlength{\columnsep}{0.5in}
  \usepackage{changepage} %allows adjustmargin

%%Bibliograpy
\usepackage{csquotes} %citations and block quotes
  \usepackage[notes,backend=biber,useibid=true]{biblatex-chicago} %citation style
  \bibliography{mena.bib} %% CHANGE THIS ON NEW DOCS
  \usepackage[dvipsnames]{xcolor}
  \usepackage[citecolor=teal,]{hyperref} %%makes urls in biblio clickable, the [ is for the ibid links]
  \hypersetup{urlcolor=RoyalBlue, colorlinks=true}
  \renewcommand{\thefootnote}{\textcolor{black}{\arabic{footnote}}}
  \renewcommand*{\bibfont}{\raggedright} %formatting Bib's font
  \usepackage[ragged,hang,bottom]{footmisc} %footnote customisation package
    \footnotemargin 0.125in %space between number and text
    \addtolength{\skip\footins}{1pc plus 5pt} %space above footnote line
    \interfootnotelinepenalty=10000 %% makes it near impossible for footnotes to be split across pages

%%Environment Styling
  \usepackage{titling} %Title Formatting
    \pretitle{\sc\begin{flushleft}\Huge}
    \posttitle{\par\end{flushleft}}
    \preauthor{\begin{flushleft}
    \Large \lineskip 0.5em}
    \postauthor{\par\end{flushleft}}
    \predate{\begin{flushleft}\large}
    \postdate{\par\end{flushleft}}
  \usepackage{sectsty} % Section Styling
    \sectionfont{\sc\raggedright\singlespace\LARGE}
    \subsectionfont{\sc\mdseries\singlespace\raggedright}
  %% Sections use Roman Numerals
    \renewcommand{\thesection}{\Roman{section}}
    \renewcommand{\thesubsection}{\thesection.\Roman{subsection}}
  %% Table of Contents Styling
    \usepackage[titles]{tocloft}
    \renewcommand{\cftsecfont}{\color{RoyalBlue}}

  \usepackage{etoolbox} %enables begin/end environment styling
    \AtBeginEnvironment{quote}{\singlespacing\small}
    \AtBeginEnvironment{abstract}{\setstretch{1.15}}
    \AtBeginEnvironment{multicols}{\setstretch{1.15}}



%%Watermarking
  %\usepackage{draftwatermark}
  %  \SetWatermarkText{D R A F T} %removed \sffamily if not XeLaTeX
  %  \SetWatermarkScale{1}


%% MAKETITLE STUFF
\title{\Huge The One-State Reality \& the Blame Game}
\author{\large Alexander Horne}
\def\course{MENA Politics} %defines the variable ``course''
  \makeatletter %this sets variables for invoking title, author &c
  \let\runauthor\@author
  \let\runtitle\@title
  \let\runCourse\course %invokes ``course''
\date{} %using the Date for now
\usepackage{nameref} %%Allows a running section name in header
  \makeatletter
  \newcommand*{\currentname}{\@currentlabelname}
  \makeatother

\usepackage{fancyhdr} %% Header/Footer stuff
  \pagestyle{fancy}
  %\chead{\runtitle}
  %\rhead{\thesection}
  %\lhead{\runCourse}
  %\setlength{\headheight}{40pt}
  \cfoot{} %keeps center empty
  \fancyhead[RE,LO]{Alexander Horne}
  \fancyhead[LE,RO]{\sc The One-State Reality}
  %\fancyhead[LE,RO]{\setmainfont{New York Small}\runauthor}
  \fancyhf[FLE,FRO]{\thepage}
  \fancyhf[FRE,FLO]{\runCourse}

\usepackage{lettrine}

\usepackage[nottoc,numbib]{tocbibind}

%%%%%%%%%%%%%%%%%%%%%
%%
%% BEGIN DOCUMENT
%%
%%%%%%%%%%%%%%%%%%%%%


\begin{document}

\maketitle

\begin{adjustwidth}{1in}{1in}
    \lipsum[2]
\end{adjustwidth}

\begin{multicols}{2}

\lettrine[lraise=0.1, nindent=0em, slope=-.5em]{V}{oici} 

\section{The Iron Wall's Success \& Failure}

As Lustick argues, there was no single discrete decision, event, or leader in Israel which is chiefly responsible for the TSS failure. He writes that the ``mechanism of unintended consequences'' over a century has made it impossible.\autocite{lustick2019paradigm} Zionism has so thoroughly succeeded at transforming Israeli politics that the no parties can move past it; this century old political movement is now confronted with challenges which its axioms are ill-suited to address. Even though Anti-semitism persists throughout MENA and indeed the West, hard right political groups within Israel have found willing allies on the Western Right as well, such as the race-baiting Trump administration and Orban's Hungary. 

Zionism is an intellectual offshoot of nineteenth century nationalist movements, coming of age in an era where new European states were created from fragmented polities united by a common language. Palestine was the most feasible location to realise a similar project for Jews, but only because the territory was administered by a single Empire -- first the Ottomans and then the British -- which meant there were no international boundaries to be dissolved. The historic ties of Jewish people to the land were not incidental, but they alone could not make the idea of a Jewish State a possible reality. Jews needed a vanguard and a plan of their own.

The ``decision not to decide'' on the refugee problem afforded successive Israeli governments with strategic latitude on a question corrosive to state sovereignty and regime legitimacy. However, even at first, both the left and right wings of early Israel understood that Palestinians would not be easily won over: on the right, Vladimir Jabotinsky admitted that he knew of no precedent ``where a country was colonized with the courteous consent of the native population.''\autocite[113]{caplan1978question} Ben-Gurion himself conceded, ``We as a nation want this country to be ours; the Arabs, as a nation, want this country to be theirs.''\autocite[80]{smith1988palestine} Even if the latter would argue in 1946 to the Anglo-American Committee of Inquiry that the conflict was a ``passing'' thing, it did not change the fact that the movement had no alternative plan for the Palestinian question aside from hoping the will to resist would falter. 

Lustick describes the Iron Wall Strategy proposed by Jabotinsky as broken down in five stages. After the construction of a rigid boundary (1), it would be definitively defended (2). After some time, the futility of open hostility would discredit Palestinian militants and elevate moderates who sought to negotiate a compromise (3). Then, the Israeli defenders would reward the moderates with a dialogue and talks (4), before finally (5) both sides would arrive at a settlement.\autocite{lustick2019paradigm} 

Needless to say, Jabotinsky was not thinking clearly. Even without hindsight, it is not difficult to foresee that Palestinian militants would give way to \textit{more extreme} militants, either more strongly motivated or by more extreme and adjacent ideologies. This plan of action broadly corresponds to the actions undertaken by the Likud governments starting with Begin, who at Camp David seemed to arrive at stage 3 with Sadat. But, rewarding moderation with international praise (ironically) discredited it within the Palestinian camp.

Even though Islamist militias such as HAMAS, Black September, and others do not represent or speak on behalf of the entirety of the Palestinian people, their emergence illustrates that defeat may dissuade state-actors from violence but not necessarily non-state groups. Indeed, it eventually became state policy to elevate the extremist minority factions over moderates so as to destabilise the Palestinian Authority and legitimise the case for indefinite military rule.

Within Jabotinsky's doctrine, the events of the 1948 war make more sense. The Palestinians sought to make the UN partition unenforceable by kicking over the proverbial apple cart; the counter-offensives pushing beyond the plan's boundaries were part of the ``definitive defence'' of the Iron Wall. The success of the Israeli military validated Jabotinsky's theory, so unsurprisingly the IDF continued to employ it, in the Suez Crisis (1956), the Six-Day War (1967), and again in the October War (1973). Each time, even though Israel secured a military victory, the defeat either hardened the resolve of the losers or led to their replacement by more extreme militants.

Within Israel's political society, a similar hardening took place: since Israel was so preponderant, what benefit is there to negotiating with Arab moderates at all? Why not seek concessions on their own terms? This question explains the apparent contradiction between Begin's accords with Sadat and his expansion of settlements in the territories.

The Oslo Process was always a long-shot, but the base on which it rest was already eroded. Within the Knesset, hard line parties opposed Rabin treating with the enemy; Arafat's PLO was already waning as the most important faction within Palestinian politics. Netanyahu's first government was not the decisive death knell to the TSS but the culmination of decades long trends.

\section{International Partners}

Part of the blame also falls on states like the United States and United Kingdom, countries which had vested interests in the region but bore little risk for supporting the state-building project. At least one can say that the governments supporting the Palestinian cause had a legitimate interest in the outcome, since they were proximal to the conflict and not oceans away. 


\end{multicols}

\nocite{kelman2018onetwo}
\nocite{lustick2019paradigm}
\nocite{kear2022wasatiyyah}
\nocite{omalley2017israel}
\nocite{scheindlin2016confederalism}
\nocite{munayyer2019there}
\nocite{gordon2012western}
\nocite{farsakh2011one}

\vfill
\pagebreak
\printbibliography

\end{document}

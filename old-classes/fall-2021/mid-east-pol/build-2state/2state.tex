\documentclass[letterpaper,12pt,twoside]{article} %define template defaults, paper size, main font, \&c.
  \usepackage[T1]{fontenc} %remove T2A to use other roman fonts, also have to remove all cyrillic text

%%Fonts \& Typesetting
  \usepackage{microtype} %pretty kerning
  \usepackage{dirtytalk} %easier to do quotation marks
  \usepackage{setspace} %setspace and setstrecth
  \usepackage{lipsum} %generates lorem ipsum
  %\usepackage{blindtext}
  \usepackage{mathptmx}
  \usepackage[super]{nth}

%%Page Geometry
\usepackage[top=1in, bottom=1in, left=0.75in, right=0.75in,headsep=0.2in]{geometry}
  \usepackage{multicol}
  \setlength{\columnsep}{0.5in}
  \usepackage{changepage} %allows adjustmargin

%%Bibliograpy
\usepackage{csquotes} %citations and block quotes
  \usepackage[notes,backend=biber,useibid=true]{biblatex-chicago} %citation style
  \bibliography{mena.bib} %% CHANGE THIS ON NEW DOCS
  \usepackage[dvipsnames]{xcolor}
  \usepackage[citecolor=teal,]{hyperref} %%makes urls in biblio clickable, the [ is for the ibid links]
  \hypersetup{urlcolor=RoyalBlue, colorlinks=true}
  \renewcommand{\thefootnote}{\textcolor{black}{\arabic{footnote}}}
  \renewcommand*{\bibfont}{\raggedright} %formatting Bib's font
  \usepackage[ragged,hang,bottom]{footmisc} %footnote customisation package
    \footnotemargin 0.125in %space between number and text
    \addtolength{\skip\footins}{1pc plus 5pt} %space above footnote line
    \interfootnotelinepenalty=10000 %% makes it near impossible for footnotes to be split across pages

%%Environment Styling
  \usepackage{titling} %Title Formatting
    \pretitle{\sc\begin{flushleft}\Huge}
    \posttitle{\par\end{flushleft}}
    \preauthor{\begin{flushleft}
    \Large \lineskip 0.5em}
    \postauthor{\par\end{flushleft}}
    \predate{\begin{flushleft}\large}
    \postdate{\par\end{flushleft}}
  \usepackage{sectsty} % Section Styling
    \sectionfont{\sc\raggedright\singlespace\LARGE}
    \subsectionfont{\sc\mdseries\singlespace\raggedright}
  %% Sections use Roman Numerals
    \renewcommand{\thesection}{\Roman{section}}
    \renewcommand{\thesubsection}{\thesection.\Roman{subsection}}
  %% Table of Contents Styling
    \usepackage[titles]{tocloft}
    \renewcommand{\cftsecfont}{\color{RoyalBlue}}

  \usepackage{etoolbox} %enables begin/end environment styling
    \AtBeginEnvironment{quote}{\singlespacing\small}
    \AtBeginEnvironment{abstract}{\setstretch{1.15}}
    \AtBeginEnvironment{multicols}{\setstretch{1.15}}



%%Watermarking
  %\usepackage{draftwatermark}
  %  \SetWatermarkText{D R A F T} %removed \sffamily if not XeLaTeX
  %  \SetWatermarkScale{1}


%% MAKETITLE STUFF
\title{\Huge The One-State Reality \& the Blame Game}
\author{\large Alexander Horne}
\def\course{MENA Politics} %defines the variable ``course''
  \makeatletter %this sets variables for invoking title, author &c
  \let\runauthor\@author
  \let\runtitle\@title
  \let\runCourse\course %invokes ``course''
\date{} %using the Date for now
\usepackage{nameref} %%Allows a running section name in header
  \makeatletter
  \newcommand*{\currentname}{\@currentlabelname}
  \makeatother

\usepackage{fancyhdr} %% Header/Footer stuff
  \pagestyle{fancy}
  %\chead{\runtitle}
  %\rhead{\thesection}
  %\lhead{\runCourse}
  %\setlength{\headheight}{40pt}
  \cfoot{} %keeps center empty
  \fancyhead[RE,LO]{Alexander Horne}
  \fancyhead[LE,RO]{\sc The One-State Reality}
  %\fancyhead[LE,RO]{\setmainfont{New York Small}\runauthor}
  \fancyhf[FLE,FRO]{\thepage}
  \fancyhf[FRE,FLO]{\runCourse}

\usepackage{lettrine}

\usepackage[nottoc,numbib]{tocbibind}

%%%%%%%%%%%%%%%%%%%%%
%%
%% BEGIN DOCUMENT
%%
%%%%%%%%%%%%%%%%%%%%%


\begin{document}

\maketitle

\begin{multicols}{2}

\lettrine[lraise=0.0, nindent=0.5em, slope=-.5em]{I}{n} December of 2017, Donald Trump announced that the US would recognise Jerusalem as the capital of Israel and relocate its embassy to the city. The decision was roundly criticised by the international community as needlessly provocative, raising worries that the possibility for a Two State Solution to the Palestine Question would be irreparably jeopardised. 

The reality is that the Two-State Solution has been deader than dead for some time. The axiomatic precepts from which it is derived bore little resemblance to reality and failed to understand the motivations and psychology of the primary actors on all sides. Five years later, there is no denying that the TSS is utterly impossible as a political project. 

But who is to blame? How did we arrive at such a ridiculous point, where a failed real-estate mogul could comfortably announce that his decision was motivated not by \textit{raison d'état} but rather by a narrow evangelical constituency who never set foot in Israel? The seeds of this catastrophic impasse were sown decades ago, and we must begin there if we want to learn anything from this debacle.

\section{The Iron Wall's Success \& Failure}

As Lustick argues, there was no single discrete decision, event, or leader in Israel which is chiefly responsible for the TSS failure. He writes that the ``mechanism of unintended consequences'' over a century has made it impossible.\autocite{lustick2019paradigm} Zionism has so thoroughly succeeded at transforming Israeli politics that the no parties can move past it; this century old political movement is now confronted with challenges which its axioms are ill-suited to address. Even though Anti-semitism persists throughout MENA and indeed the West, hard right political groups within Israel have found willing allies on the Western Right as well, such as the race-baiting Trump administration and Orban's Hungary. 

Zionism is an intellectual offshoot of nineteenth century nationalist movements, coming of age in an era where new European states were created from fragmented polities united by a common language. Palestine was the most feasible location to realise a similar project for Jews simply because the territory was administered by a single Empire -- first the Ottomans and then the British -- meaning no international boundaries needed to be dissolved. The historic ties of Jewish people to the land were not incidental at all, but they alone could not make the idea of a Jewish State a possible reality. Jews needed a vanguard and a plan of their own.

The ``decision not to decide'' on the refugee problem afforded successive Israeli governments with strategic latitude on a question corrosive to state sovereignty and regime legitimacy. However, even at first, both the left and right wings of early Israel understood that Palestinians would not be easily won over: on the right, Vladimir Jabotinsky admitted that he knew of no precedent ``where a country was colonized with the courteous consent of the native population.''\autocite[113]{caplan1978question} Ben-Gurion himself conceded, ``We as a nation want this country to be ours; the Arabs, as a nation, want this country to be theirs.''\autocite[80]{smith1988palestine} Even if the latter would argue in 1946 to the Anglo-American Committee of Inquiry that the conflict was a ``passing'' thing, it did not change the fact that the movement had no alternative plan for the Palestinian question aside from hoping the will to resist would falter. 

Lustick describes the Iron Wall Strategy proposed by Jabotinsky as broken down in five stages. After the construction of a rigid boundary (1), it would be definitively defended (2). After some time, the futility of open hostility would discredit Palestinian militants and elevate moderates who sought to negotiate a compromise (3). Then, the Israeli defenders would reward the moderates with a dialogue and talks (4), before finally (5) both sides would arrive at a settlement.\autocite{lustick2019paradigm} 

Needless to say, Jabotinsky was not thinking clearly. Even without hindsight, it is not difficult to foresee that Palestinian militants would give way to \textit{more extreme} militants, either more strongly motivated or by more extreme and adjacent ideologies. This plan of action broadly corresponds to the actions undertaken by the Likud governments starting with Begin, who at Camp David seemed to arrive at stage 3 with Sadat. But, rewarding moderation with international praise (ironically) discredited it within the Palestinian camp.

Even though Islamist militias such as HAMAS, Black September, and others do not represent or speak on behalf of the entirety of the Palestinian people, their emergence illustrates that defeat may dissuade state-actors from violence but not necessarily non-state groups. Indeed, it eventually became state policy to elevate the extremist minority factions over moderates so as to destabilise the Palestinian Authority and legitimise the case for indefinite military rule.

Within Jabotinsky's doctrine, the events of the 1948 war make more sense. The Palestinians sought to make the UN partition unenforceable by kicking over the proverbial apple cart; the counter-offensives pushing beyond the plan's boundaries were part of the ``definitive defence'' of the Iron Wall. The success of the Israeli military validated Jabotinsky's theory, so unsurprisingly the IDF continued to employ it, in the Suez Crisis (1956), the Six-Day War (1967), and again in the October War (1973). Each time, even though Israel secured a military victory, the defeat either hardened the resolve of the losers or led to their replacement by more extreme militants.

Within Israel's political society, a similar hardening took place: since Israel was so preponderant, what benefit was there to negotiating with Arab moderates at all? Why not negotiate on their own terms? This question explains the apparent contradiction between Begin's accords with Sadat and his expansion of settlements in the territories.

The Oslo Process was always a long-shot, but the base on which it rest was already eroded. Within the Knesset, hard line parties opposed Rabin treating with the enemy; Arafat's PLO was already waning as the most important faction within Palestinian politics. Netanyahu's first government was not the decisive death knell to the TSS but the culmination of decades long trends.

This statist interpretation of the Shoah leads Israeli statesmen and voters to be suspicious of outsiders who insist a non-zero sum outcome is possible (let alone desirable). In the world's only Jewish State, one which exists to protect the Jewish people, who are gentile allies to advise Israel on how they might seek peace with its neighbours? This means that Likudniks are ideologically incapable of absorbing evidence that their own policies have made the TSS untenable or impossible.

\section{The History of the Two-State Solution}

Like Zionism, a two-state partition of Palestine arose from a particular historical context and ideological tradition from the past. The TSS is essentially Wilsonian in character -- believing that two democratic nations can coexist across definite borders -- and first emerged under the auspices of Wilsonian Liberalism's first confederation, the League of Nations. 

This idealism stands in stark contrast with Israel's military adventurism in defence of the Iron Wall. Lustick, writing about Zionism as a ``degenerative research project,'' argues that this mounting contradiction between liberal self-determination for Israel and Israel's actions against the peace, is the essence of the problem. The TSS only works within a liberal idealist paradigm, but the \textit{reality} is that Israel is and always has been a \textit{settler colonialist} state. As discussed earlier, Zionists on both left and right were aware of this fact; it is only in the intervening years that Palestinian and Israeli doves came to believe that a plan drafted by a previous Empire could ever work.

\section{International Partners}

Part of the blame also falls on states like the United States and United Kingdom, countries which had vested interests in the region but bore little risk for supporting the state-building project. At least one can say that the governments supporting the Palestinian cause had a legitimate interest in the outcome, since they were proximal to the conflict and not oceans away. 

The Bush Sr. administration was the last chance the TSS had to really succeed, but several factors conspired to make this a long shot as well. First, Bush Sr. entered office after 8 years of Republican rule in the US; it was unlikely that he would win reëlection in 1992 \textit{and} maintain Congress (neither happened). The Labour interregnum of the '90s never seriously put an end to settlement expansion in the West Bank, and the Clinton Administration refused to exert pressure to follow Oslo. Thus, the Likud strategy was validated before Netanyahu rode the party to power.

Alarmists worried that a point of no return was reached as early as 1987; by this point, the Two-State Solution is deader than dead. One might go further and argue that the TSS never bore any theoretical relation to the facts on the ground anyway. Who is to blame? All sides bear a share of the guilt, but the one group with the most latitude for action since 1917 has been Zionist statists. Only one side has the organisational means and legal cover to construct settlements on the other's turf. No matter how foolhardy a TSS would have been, the opportunity to pursue it as an end was repeatedly denied by the right wing of political Zionism, culminating in the rise of Likudism in Israel. This is doubly certain since the end of hostilities with Egypt; the decades since have witnessed diminishing Arab solidarity with Palestinians, to the point that only non-state actors and international pariahs like Tehran and Damascus oppose Israel militarily. The Likudniks have succeeded in obtaining neutrality from their Arab neighbours, and so they alone get to own the results.

To be sure, Likud's success was impossible without the bizarre and unconditional support of the United States, and therefore they bear, in this author's estimation, the second largest share of the blame. Presidents of both parties and congress after congress enabled the rightward drift of Israel's politics, either because it suited a short-term political end or because the short-term consequences were intolerable. That the long-term consequences were amenable to American Imperialism -- Israel is essentially an unsinkable aircraft carrier near the most critical oilfields in the world -- should not be ruled out as entirely accidental either. 

\section{The Palestinian Side}

It takes two to tango, and Israel could not have killed the two state solution on its own. This author's opinion is that the first cause was British imperialism, against which the Palestinians were forced to react.

But one should not undersell the contribution which neighbouring states also made; resistance to Zionism became militarised as neighbouring Arab countries attempted to link the dispute to Arab solidarity. But leaders like Nasser and Hafez al-Assad have been dead for a long time, and since their deaths most of the MENA has accomodated itself to Israel's presence -- to know positive end vis-a-vis the Palestine Question.

The only groups which hold out are non-state insurgent groups or proxies of other states, and they run the gamut of supporting a Two State Solution to demanding the departure of all Jews. Their stances were not invented in vacua but developed in response to political conditions. Israel and the West, as the most powerful and influential factions in this region, effectively determined those conditions, especially as the Arab states came to accept the Iron Wall.

Take for example HAMAS, the most significant resistance force in the Occupied Palestinian Territories. As Kear argues, the group may not be a pleasant one but its success within Palestinian elections moderated their rhetoric, positions and goals.\autocite{kear2022wasatiyyah} The leadership since 2005 has adjusted its demands to call for sovereignty within the OPT -- not the return of all Israeli territory -- and has declared its support for UNSCR 242, which would mean a \textit{de facto} recognition of Israeli. These gestures matter very little; Israel bombed the territories as punishment for Palestinians voting the ``wrong'' way. In the ghettos where Palestinians are confined, Israel was historically content to allow HAMAS to play the role of a state: creating mutual aid networks, collecting martyrdom funds, in essence building legitimacy. Only after HAMAS could meaningfully challenge Fatah's authority did Israel react.

\section*{conclusion}

The Two-State Solution was impossible from the start. Early Zionist leaders were aware of this fact and Likud's leaders are either in denial or engaging in a double dialogue to conceal their awareness. The high-water mark for its practicality was 1992, when neoconservatives like Rabin and Bush could stake their credibility without being condemned as sellouts. It did not matter to the hard-right in Israel; AIPAC fought to defeat Bush Sr. and Rabin met his end with a bullet. 

The one statesmen more than any other who is responsible for the end of a TSS is Menachem Begin (an argument could be made for his political disciple, Netanyahu). From his beginnings in the Mandate years with Irgun to the Lebanese Civil War, Begin used every tool he could to discredit voices for peace, undermine moderates, and gin up hardline nationalism.

Every imperial power in the Middle East shares the blame for enabling or permitting the situation. The US in particular deserves castigation, as it wields its veto in the UNSC to support Israel unconditionally. So much for the responsibility to protect.

\end{multicols}

\nocite{kelman2018onetwo}
\nocite{lustick2019paradigm}
\nocite{kear2022wasatiyyah}
\nocite{omalley2017israel}
\nocite{scheindlin2016confederalism}
\nocite{munayyer2019there}
\nocite{gordon2012western}
\nocite{farsakh2011one}

\vfill
\pagebreak
\printbibliography

\end{document}

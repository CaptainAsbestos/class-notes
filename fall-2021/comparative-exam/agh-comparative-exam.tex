\documentclass[12pt]{article}

\usepackage{amsmath}

\usepackage{graphicx}

\usepackage{hyperref}

\usepackage[utf8]{inputenc}

\RequirePackage[
    backend=biber,
    style=apa,
    citestyle=authoryear-ibid,
    autocite=footnote,
]{biblatex}

\addbibresource{agh-comparative-exam.bib}

\title{Final Exam}

\author{Alex Horne}

\date{2022}

\begin{document}

\maketitle

\section*{Group A}

\renewcommand{\abstractname}{Question 1}
\begin{abstract}
    \textit{
        Comparative politics focuses extensively on democracy. Research grapples with the concept, the types of democracy, the emergence and stability of democracy, and, most recently, the crisis of democracy. Based on our reading, particularly the Przeworski book, what are the critical ingredients in a democracy, what is the nature of the “crisis” of democracy today, and what factors have contributed to the crisis of democracy in recent years?
    }
\end{abstract}

Przeworski employs a minimalist concept of democracy, writing that "democracy is a political arrangement in which people select governments through elections and have a reasonable possibility of removing incumbent governments they do not like." \autocite[5]{przeworski2019crises} He argues that the other trappings of liberal democracy are downstream of this narrow conceptualisation, that they spring from it as natural consequences of this electoralist minimum. He prefers this definition because it makes it possible to operationalise and observe elements of democracy, and with his observations he makes the case that there is indeed a political crisis in the democratic world.

Even though Przeworski is joined by several prominent theorists and researchers in defining democracy narrowly, we have also encountered other, more subtantive definitions.

also, the happy medium of elections being "important enough" to be worth voting in but also not so important that they pose an existential threat to the losers

Schmitter and Karl (1991) write that "democracy is a system of governance in which rulers are held accountable for their actions in the public realm by citizens, acting indirectly through the competition and cooperation of their elected representatives." They also note that most social scientists employ some variant of economist Joseph Schumpeter's definition, "that institutional arrangement for arriving at political decisions in which individuals acquire the power to decide by means of a competitive struggle for the people's vote." Schmitter and Karl's approach emphasises the functional aspects of democracy -- what it does and how it does it -- rather than what it looks like. This comes as little surprise, for the diversity of the world's cultures & societies would necessarily produce wide diversities between the democracies governing them. But Schmitter and Karl take pain to note what democracy \textit{is not}. Democracy is not a perfectly stable or orderly political structure; it does not guarantee economic "efficiency" or even development. Democracies will almost certainly be administratively inefficient if their charters balance power appropriately between political institutions. Democracies are also relatively more transparent and open to deliberation, but a regime of civil liberties only rarely accompanies a regime of liberal market capitalism. 

I mention these extra aspects of Schmitter and Karl's work because they form the basis, in part, on which Przezworski's crisis of democracy sits. On the liberal side of US politics, Fukuyama's prophesied End of History enamoured centrists and Third-Way democrats. Many of them were captured by the notion that the dysfunction of Western Democracy during the Cold War stemmed \textit{from} the Cold War; once free-market neoliberalism became entrenched around the world, it was hoped, societies governed by authoritarians would democratise naturally (the modernisation hypothesis). These optimists should have heeded Schmitter's and Karl's warning that open societies exist in tension, not conflict or cooperation, with open markets. Przezworski alludes to this fact in his own work: the vanishing hope of upward mobility in the globalised economy leads citizens and subjects alike to tolerate illiberalism. Indeed, for what good are civil liberties if the first freedom you have is to starve?

accountability; glasius sees authoritarianism in opposition to democracy \textit{because} it erodes accountability

\section*{Group B}

relate varshner's work on populism and the right wing to przeworski's thoughts on upward mobility

civil war likelihood (walter) and przeworski's pessimism on democratic crises

Rational Choice theory as a conceptual throughline (acemoglu/robinson, kohli, laitin (incentives for joining groups) )

RCT interfacing with corruption?? interesting idea

"institutions" shaping the incentives which guide people's rational choices between alternatives (acemoglu)

\section*{Group C}

Question 6 -- relate the corruption discussion to the Foreign investment decision

\printbibliography

\end{document}

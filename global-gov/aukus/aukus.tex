\documentclass[letterpaper,12pt,twoside]{article} %define template defaults, paper size, main font, \&c.
  \usepackage[T1]{fontenc} %remove T2A to use other roman fonts, also have to remove all cyrillic text

%%Fonts \& Typesetting
   \usepackage{fontspec}%for XeLaTex, any OTF or TTF font installed will work
    \setromanfont{New York Medium} %roman often overrides main font
    %\setromanfont{Hack}
    \setsansfont{Frutiger LT Std 55 Roman}
    %\setsansfont{Ubuntu}
    \setmonofont{Hack}
  %%\usepackage's are for pdfLaTeX
    %\usepackage{CormorantGaramond}
    %\usepackage{lmodern}
    %\usepackage{mathptmx} %Times new roman
    %\usepackage{palatino}
    %\usepackage{helvet} %Helvetica
    %\usepackage{ebgaramond}
    %\usepackage{garamondlibre}
    %\def\ttdefault{lmtt} %latin modern mono
    %\renewcommand{\familydefault}{\sfdefault} %switches default to sans serif
  \usepackage{microtype} %pretty kerning
  \usepackage{setspace} %setspace and setstrecth
  \usepackage{lipsum} %generates lorem ipsum
  %\usepackage{blindtext}

%%Page Geometry
\usepackage[top=1in, bottom=1in, left=0.75in, right=0.75in,headsep=0.2in]{geometry}
  \usepackage{multicol}
  \setlength{\columnsep}{0.5in}

%%Bibliograpy
\usepackage{csquotes} %citations and block quotes
  \usepackage[notes,backend=biber]{biblatex-chicago} %citation style
  \bibliography{japan} %% CHANGE THIS ON NEW DOCS
  %\hypersetup{urlcolor=blue, colorlinks=true}
  \renewcommand*{\bibfont}{\raggedright} %formatting Bib's font
  \usepackage[ragged,hang,bottom]{footmisc} %footnote customisation package
    \footnotemargin 0.125in %space between number and text
    \addtolength{\skip\footins}{1pc plus 5pt} %space above footnote line
    \interfootnotelinepenalty=10000 %% makes it near impossible for footnotes to be split across pages
  \usepackage{hyperref} %%makes urls in biblio clickable

%%Environment Styling
  \usepackage{titling} %Title Formatting
    \pretitle{\sc\begin{flushleft}\Huge}
    \posttitle{\par\end{flushleft}}
    \preauthor{\begin{flushleft}
    \Large \lineskip 0.5em}
    \postauthor{\par\end{flushleft}}
    \predate{\begin{flushleft}\large}
    \postdate{\par\end{flushleft}}
  \usepackage{sectsty} % Section Styling
    \sectionfont{\sc\raggedright\singlespace\LARGE\color{blue}}
    \subsectionfont{\mdseries\singlespace\raggedright}

  \usepackage{etoolbox} %enables begin/end environment styling
    \AtBeginEnvironment{quote}{\singlespacing\small}
    \AtBeginEnvironment{abstract}{\setstretch{1.15}}
    \AtBeginEnvironment{multicols}{\setstretch{1.15}}

%%Watermarking
  %\usepackage{draftwatermark}
  %  \SetWatermarkText{D R A F T} %removed \sffamily if not XeLaTeX
  %  \SetWatermarkScale{1}


%% MAKETITLE STUFF
\title{Non-Proliferation in the Indo-Pacific}
\author{Alexander Horne}
\def\course{PS-6400} %defines the variable ``course''
  \makeatletter %this sets variables for invoking title, author &c
  \let\runauthor\@author
  \let\runtitle\@title
  \let\runCourse\course %invokes ``course''
\date{} %using the Date for now
\usepackage{nameref} %%Allows a running section name in header
  \makeatletter
  \newcommand*{\currentname}{\@currentlabelname}
  \makeatother

\usepackage{fancyhdr} %% Header/Footer stuff
  \pagestyle{fancy}
  %\chead{\runtitle}
  %\rhead{\thesection}
  %\lhead{\runCourse}
  %\setlength{\headheight}{40pt}
  \cfoot{} %keeps center empty
  \fancyhead[RE,LO]{\runauthor}
  \fancyhead[LE,RO]{\sc Non-Proliferation in the Indo-Pacific}
  %\fancyhead[LE,RO]{\setmainfont{New York Small}\runauthor}
  \fancyhf[FLE,FRO]{\thepage}
  \fancyhf[FRE,FLO]{\runCourse}

\usepackage{lettrine}
\usepackage{graphicx}
\usepackage{xcolor}

\usepackage[nottoc,numbib]{tocbibind}

%%%%%%%%%%%%%%%%%%%%%
%%
%% BEGIN DOCUMENT
%%
%%%%%%%%%%%%%%%%%%%%%


\begin{document}
%\begin{CJK*}%{UTF8}{gbsn} %uncomment to reenable chinese
\begin{titlepage}
  \maketitle
  %\vfill
   %\begin{center}
      %\includegraphics[width=1.5in]{care-logo.png}
    %\end{center}
\vfill

\renewcommand{\abstractname}{\sc\large Executive Summary}
\begin{abstract}

    On the fifteenth of September, 2021, the United States, Australia, and the United Kingdom announced a trilateral security pact for the Indian and Pacific Ocean Regions. AUKUS, as the pact is known, is part of a decades long ``pivot'' towards Asia to counter the growing influence of Mainland China. As part of the agreement, the UK and the US would help Australia acquire nuclear-power submarines, raising the spectre of renewed nuclear proliferation in one of the most security critical regions of the world. This paper argues that, even within the confines of non-proliferation protocols, the political precedent which AUKUS establishes opens the door for similar technology sharing on all sides, which serves to undermine a historically effective international security regime.

\end{abstract}
%\smallskip

\end{titlepage}

\tableofcontents
  \vfill
  \pagebreak

\begin{multicols}{2}

\section*{Introduction}

\section{Nuclear Power}

The United States remains the only state in the world to have used nuclear weapons during war -- twice against civilian populations. At the time, the bomb was heralded as an efficient and modern weapon, able to kill thousands with far less hazard to US pilots. Since then, a small handful of other states have developed their own bombs: the Soviet Union (and thus its successor, the Russian Federation), the United Kingdom, France, China, Pakistan, India, and North Korea. During the Cold War, several other nations hosted the nuclear weapons of their allies, such as Cuba, Germany, Italy, Turkey, and parts of the former USSR. South Africa relinquished its nuclear weapons with the end of Apartheid; Israel denies possessing nuclear weapons, but it is speculated that they do (after all, an open secret is still a deterrent).  Lastly, there are countries which have attempted to develop their own nuclear weapons and have so far been prevented from doing so. Notable among these are Iran and Qaddafist Libya, the latter abandoning its plans in the mid-2000s.

Nuclear energy, on the other hand, is a permitted use for fission technology under international atomic protocol. Countries on every continent either current operate or plan on building nuclear reactors for civilian energy purposes. Nuclear reactors aboard navies allow ships to sail for years at a time without refuelling, requiring regular maintenance checkups. What all reactors have in common is the high cost of decommissioning spent fuel and worn out reactors.

US nuclear designs leveraged their pre-existing ability to generate weapons grade uranium (above 90\% U-235) for their reactors; this is well above the needed quality for a reactor. France, despite also possessing the means to enrich weapons-grade material, has explored the use of low-enriched uranium (ca. 5\% U-235) for its own naval reactors.

\section{A New Cold War?}

There have already been attempts to make a ``Indo-Pacific'' NATO in the 20th Century. In East Asia, there was the Southeast Asia Treaty Organization (SEATO) formed in 1954. Its members included European colonial powers (France, UK) and countries colonised by those powers (Australia, New Zealand, Pakistan, Philippines, Thailand, United States). Vietnam, Cambodia, and Laos were protocol members while still under French control. This organisation dissolved in 1977, having mostly failed in its purpose of ``containing'' communism in East Asia.

A similar fate befell the Central Treaty Organisation (CENTO), formed in 1955 and dissolving in 1979. Its members included the UK, Iran, Iraq, Pakistan, and Turkey. Its purpose was similar to NATO and SEATO -- containing the Soviet Union -- but its location, stretching across Southwest Asia, meant that the more pressing concern would be Arab-Israeli conflict and anti-colonial struggles. The regime changes in Iraq and Iran and the Turkish invasion of British controlled-Cyprus underlined the lack of political cohesion between member-states, leading to its inevitable dissolution.

From an Atlantic point of view, these alliances were too incohesive to last. Indeed, the question is why NATO survived while the others dissolved. What all these treaty organisations have in common is that they were negotiated with minimal input from the publics of the respective member states -- a fact reflected in the surprise towards AUKUS's announcement.

The new balance of power in the Indo-Pacific is markedly different from that of 1945.

France paragraph
New Zealand paragraph


\section{Australian Nuclear Power}

As the only member of AUKUS not possessing nuclear weapons, the question remains how the country will develop the means to build and maintain a conventionally armed nuclear-powered navy. The island nation relies extensively on carbon for domestic energy, but it is the world's third largest producer of uranium.

\section{International Precedent}

\end{multicols}
  \pagebreak


\printbibliography[heading=bibintoc,title=Bibliography]


%\end{CJK*} %uncomment to reenable chinese
\end{document}
